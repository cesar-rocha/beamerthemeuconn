\documentclass{beamer}

\usepackage[utf8]{inputenc}
\usepackage[T1]{fontenc}

\useinnertheme{uconn}
\useoutertheme[bgleaf,campus=health,orient=stacked]{uconn}
%\useoutertheme[footline=authorinstitute,subsection=true]{miniframes}
%\useoutertheme{sidebar}
\usecolortheme{default}
\usefonttheme{default}

\title[Beamer inner and outer themes for UConn]{Beamer inner and outer themes\\ for the University of Connecticut}
\subtitle{part of a suite of themes}
\author[Cory Brunson]{Jason Cory Brunson, PhD}
\institute[UConn Health]{Center for Quantitative Medicine\\ University of Connecticut School of Medicine}
\date{\today}

\setlength{\parskip}{.5em}

\begin{document}


\begin{frame}
\titlepage
\end{frame}


\begin{frame}{Table of contents}
\tableofcontents
\end{frame}


\section{Layout}


\begin{frame}[fragile]{Background}

The background image is a faded navy blue oak leaf, stored as a PNG.
It is activated by the \verb|bgleaf| option:

\begin{verbatim}
\useoutertheme[bgleaf]{uconn}
\end{verbatim}

Like the others used in this theme, the image comes from the \href{https://brand.uconn.edu/downloads/logos/}{bulk logos download at UConn Brand Standard}.

\end{frame}


\begin{frame}[fragile]{Logo}

The woodmark used in the footer can be determined by the \verb|logo| option, or it will be implicitly understood when values are passed to the \verb|campus| or \verb|orient| option:

\begin{verbatim}
\useoutertheme[logo]{uconn}
\useoutertheme[campus=health,orient=stacked]{uconn}
\end{verbatim}

\begin{itemize}
\item \verb|campus| understands the values \verb|health| (UConn Health), \verb|averypoint|, \verb|hartford|, \verb|stamford|, \verb|waterbury|, and \verb|none|.
\item \verb|orient| understands \verb|side|, \verb|stacked|, and \verb|none|, but \verb|none| is changed to \verb|side| when a campus is specified.
\end{itemize}
Passing other values is equivalent to passing no value.

\end{frame}


\begin{frame}[fragile]{Symbols}

The classic (black) oak leaf is encoded as a symbol \verb|\oakleaf|, and can be used in mathematical environments, e.g.
\[\oakleaf^{\mathbb{C}}=\oakleaf\otimes_{\mathbb{R}}\mathbb{C}\text,\]
or in text (\oakleaf).

An inverted-color oak leaf is encoded as \verb|\oakleafbox| (see the slide on claims and proofs).

\end{frame}


\section{Environments}


\subsection{Text environments}


\begin{frame}{Blocks}

\begin{block}{Block}
This is a text block, formatted as in the default template.
\end{block}

\begin{alertblock}{Alert block}
This is an alert text block, also formatted by default.
\end{alertblock}

More may be done in future to customize the block environments. Suggestions are welcome!

\end{frame}


\subsection{Mathematical environments}


\begin{frame}[fragile]{Definitions and examples}

\begin{definition}
A {\bfseries definition} block is by default formatted like a text block.
\end{definition}

Text formatting like \verb|\bfseries| (boldface) can be called within blocks.

\begin{example}
An \emph{example} is also a block, formatted a bit differently.
\end{example}

\end{frame}


\begin{frame}[fragile]{Claim and proof environments}

\begin{theorem}
Claim blocks, including \verb|lemma|, \verb|theorem|, and \verb|corollary|, use slant-shaped font (\verb|\slshape|) to emphasize their contents.
\end{theorem}

\begin{proof}
This proof block shows that the customary open square tombstone has been changed to a UConn oak leaf (white inside a black square).
\end{proof}

\begin{corollary}
The box oak leaf can also be used inline: \oakleafbox
\end{corollary}

\end{frame}


\section{Thanks}


\begin{frame}{Acknowledgments}

The \href{http://tug.ctan.org/macros/latex/contrib/beamer/doc/beameruserguide.pdf}{Beamer class users guide} and Thierry Masson's Beamer \href{http://www.cpt.univ-mrs.fr/~masson/latex/Beamer-appearance-cheat-sheet.pdf}{cheat sheet} were indispensable aides to building this theme, as were several helpful discussions on \href{https://tex.stackexchange.com/}{the \TeX--\LaTeX Stack Exchange}.

See the \href{http://brand.uconn.edu/}{UConn Brand Standards website}, in particular the \href{http://brand.uconn.edu/resources/powerpoint-templates/}{PowerPoint templates}, for more detail on the images and layouts used here.

Please send feedback to Cory at \href{mailto:brunson@uchc.edu}{\tt brunson@uchc.edu}.

\end{frame}


\end{document}